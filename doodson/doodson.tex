\documentclass[11pt, oneside]{article}   	% use "amsart" instead of "article" for AMSLaTeX format
\usepackage{geometry}                		% See geometry.pdf to learn the layout options. There are lots.
\geometry{letterpaper}                   		% ... or a4paper or a5paper or ... 
%\geometry{landscape}                		% Activate for rotated page geometry
%\usepackage[parfill]{parskip}    		% Activate to begin paragraphs with an empty line rather than an indent
\usepackage{graphicx}				% Use pdf, png, jpg, or eps§ with pdflatex; use eps in DVI mode
								% TeX will automatically convert eps --> pdf in pdflatex		
\usepackage{amssymb}

\title{Brief Article}
\author{The Author}
%\date{}							% Activate to display a given date or no date

\begin{document}
\maketitle

\section{Doodson Numbers}
A. T. Doodson \cite{Doodson1928} used a compact method of describing tide generating forces. Effectively, something of a quantum number description -- numbers of cycles per day, month, year, lunar node, lunar perigee, precession cycles, these being about 1 cpd, 1 cpm, 1 cpy, 1 cycle per 8.85 years [nnn], 1 cycle per 18.6 years [nnn], and 1 cycle per 26,000 years [nnn].

Doodson's choices of which month and year, ... 

I'll take Thomson, 1995, given the emphasis on earth sun distance, and take the year as the anomalistic year, 365.25693 NNN days \cite{Thomsonanomalistic} versus the tropical year of 365.2422 days \cite{almanac2001}. The splitting between the two amounts to 1 cycle per NNNN years.

For the problem at hand, the sun (day, year, precession) and moon (month, node, perigee) are not the only bodies to consider. Given the inverse cube effect of tidal forces, both Venus (0.815 M$_{earth}$, 0.3 AU minimum distance NNN) and Jupiter (318 M$_{earth}$, 4.2 AU minimum distance) have appreciable effect. For completeness, Mars, Saturn, Uranus and Neptune are included with Venus and Jupiter, all by way of their sidereal periods (length of time to regain the same position relative to the fixed stars.)

The expanded 'Doodson' numbers are, then, for cycles from the sun (day, year, precession) and moon (month, node, perigee) , and planets (Mercury to Neptune; Pluto and asteroids omitted on basis of mass and distance).

table\\
planets, mass, rmin, m/rmin**3\\

Even limiting to integral combinations of the 13 periods (3 solar, 3 lunar, 1 each for 7 planets), it becomes trivial to match \textit{some} combination of periods to any given target.  Two considerations help limit the field. First, there should be some physical meaning to the combination. In particular, as will be shown, the earth-sun distance varies with that period. Second, that the combinations involve small integers.





\end{document}  

